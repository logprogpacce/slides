\documentclass[pdf]{beamer}
\mode<presentation>{}

\usepackage{lipsum}
\usepackage[utf8]{inputenc}
\usepackage[T1]{fontenc}
\usetheme{Warsaw}
\usepackage{graphicx}

\usebackgroundtemplate{\includegraphics[width=\paperwidth, height=\paperheight]{images/background.png}}

\title{Lógica de Programação}
\subtitle{0 - Aprendizagem Cooperativa}
\author{Fco. Iago S. Mendes}

\begin{document}

\begin{frame}
	\titlepage
\end{frame}

\begin{frame}
	\frametitle{Aprendizagem Cooperativa}
	\pause
	\begin{itemize}
		\item O que é APCOOP?
		\pause
		\item Como será abordada?
		\pause
		\item Duvidas?
	\end{itemize}
\end{frame}

\begin{frame}
	\frametitle{Deveres de todos}
	\pause
	\begin{itemize}
	\item Presença em pelo menos 75\% dos encontros
	\pause
	\item Responsabilidade com as atividades desenvolvidas
	\pause
	\item Contrato de convivência
	\end{itemize}
\end{frame}

\begin{frame}
    \frametitle{Considerações finais}
    Agora que todos ja têm uma idéia de o que é a APCOOP e como ela será abordada dentro da célula e fizemos o contrato de convivência e decidido como serão os encontros da célula vamos começar!
\end{frame}

\end{document}
